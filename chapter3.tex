\chapter{Design}
% Design dell'architettura del sistema e delle interfacce utente.

\section{Frontend}

% https://www.usability.gov/what-and-why/user-centered-design.html#:~:text=The%20User%2Dcentered%20design%20(UCD,basis%20for%20many%20UCD%20methodologies.

% https://www.interaction-design.org/literature/topics/personas#:~:text=Personas%20are%20fictional%20characters%2C%20which,%2C%20experiences%2C%20behaviors%20and%20goals.

Ascent Essentials è un E-Commerce specializzato rivolto agli appassionati di attività sportive in montagna, quindi l'interfaccia utente è stata progettata con attenzione per riflettere la natura avventurosa e dinamica della clientela.

Lo \textit{User-Centered Design} fa parte della filosofia di Ascent Essentials. Ogni aspetto dell'interfaccia e dell'esperienza utente è stato progettato con l'obiettivo di soddisfare le esigenze e le aspettative degli appassionati di attività sportive in montagna:

\begin{itemize}
    \item \textbf{Ricerca utente approfondita}: Prima di iniziare il processo di progettazione, è stata condotta una ricerca approfondita sugli utenti target, compresi gli appassionati di arrampicata, alpinismo e trekking. Sono state utilizzate tecniche come interviste, sondaggi e analisi dei dati per comprendere le esigenze, i comportamenti e le preferenze degli utenti.
    \item \textbf{Coinvolgimento attivo degli utenti}: Gli utenti sono stati coinvolti attivamente durante tutto il processo di progettazione attraverso sessioni di feedback e test utente.
    \item \textbf{Creazione di Personas}: Sono state create Personas rappresentative degli utenti target (come Luca l'Alpinista e Sofia la Trekker) per focalizzare la progettazione su utenti specifici con esigenze e obiettivi distinti. Le Personas sono diventate guide chiave per valutare ogni aspetto del design, assicurando che soddisfi le esigenze di utenti reali.
    \item \textbf{Test iterativi}: L'approccio di progettazione ha incorporato cicli iterativi di test utente per identificare e correggere rapidamente eventuali problemi o inefficienze nell'interfaccia utente. Ogni fase del design è stata seguita da test utente, permettendo un'evoluzione continua basata sul feedback reale.
    \item \textbf{Design responsivo}: L'interfaccia è stata progettata per essere totalmente responsiva, garantendo un'esperienza utente coerente su diverse piattaforme e dispositivi. L'utilizzo di design fluido e layout adattivi assicura che Ascent Essentials sia accessibile su dispositivi mobili, tablet e desktop.
    \item \textbf{Accessibilità universale}: L'attenzione è stata posta sulla progettazione di un'interfaccia accessibile, rispettando le linee guida WCAG per garantire che tutti gli utenti, indipendentemente dalle loro capacità, possano utilizzare l'applicazione. Test automatici e manuali sono stati eseguiti per assicurare che l'applicazione rispetti gli standard di accessibilità.
    \item \textbf{Approccio empatico}: Gli sviluppatori hanno adottato un approccio empatico, cercando di comprendere le emozioni e le prospettive degli utenti per migliorare la qualità dell'esperienza utente complessiva. Le decisioni di design sono state guidate non solo da dati oggettivi, ma anche da una comprensione empatetica delle esigenze e dei desideri degli utenti.
\end{itemize}

\subsection{Personas}

\textbf{Persona 1: Luca l'Alpinista}

\textbf{Demografia:}
\begin{itemize}
    \item Età: 35 anni
    \item Occupazione: Insegnante di arrampicata professionista
    \item Posizione: Trento, Italia
\end{itemize}

\textbf{Caratteristiche:}
\begin{itemize}
    \item Luca è un alpinista esperto e appassionato che trascorre la maggior parte del suo tempo in montagna.
    \item Cerca attrezzature tecniche di alta qualità per affrontare le sfide dell'alpinismo.
    \item Desidera ricevere consigli da esperti e leggere recensioni dettagliate prima di acquistare.
\end{itemize}

\textbf{Obiettivi:}
\begin{itemize}
    \item Trovare attrezzature tecniche specifiche per le sue scalate.
    \item Ricevere consigli da esperti sulla giusta attrezzatura.
    \item Essere ispirato da racconti di avventure simili.

\end{itemize}

\textbf{Persona 2: Sofia la Trekker}

\textbf{Demografia:}
\begin{itemize}
    \item Età: 28 anni
    \item Occupazione: Fotografa naturalistica
    \item Posizione: Barcellona, Spagna
\end{itemize}

\textbf{Caratteristiche:}
\begin{itemize}
    \item Sofia è una trekker appassionata che ama esplorare nuovi sentieri e paesaggi naturali.
    \item Cerca attrezzature leggere e funzionali per rendere più piacevoli le sue escursioni.
    \item Desidera condividere le sue esperienze con una comunità di appassionati.
\end{itemize}

\textbf{Obiettivi:}
\begin{itemize}
    \item Esplorare facilmente una vasta gamma di prodotti adatti al trekking.
    \item Ricevere consigli su accessori pratici e leggeri.
    \item Condividere esperienze e consigli con altri appassionati.

\end{itemize}

\textbf{Persona 3: Elena la Principiante}

\textbf{Demografia:}
\begin{itemize}
    \item Età: 24 anni
    \item Occupazione: Studentessa universitaria
    \item Posizione: Firenze, Italia
\end{itemize}

\textbf{Caratteristiche:}
\begin{itemize}
    \item Elena è nuova nell'alpinismo e cerca attrezzature di qualità adatte ai principianti.
    \item Ha un budget limitato ma desidera investire in prodotti duraturi.
    \item Cerca un'esperienza di acquisto intuitiva e facile.
\end{itemize}

\textbf{Obiettivi:}
\begin{itemize}
    \item Trovare attrezzature adatte ai principianti.
    \item Risparmiare senza compromettere la qualità.
    \item Apprendere più cose sull'alpinismo attraverso l'accesso a risorse educative.

\end{itemize}

\textbf{Persona 4: Marco l'Appassionato di Escursioni}

\textbf{Demografia:}
\begin{itemize}
    \item Età: 40 anni
    \item Occupazione: Manager aziendale
    \item Posizione: Monaco di Baviera, Germania
\end{itemize}

\textbf{Caratteristiche:}
\begin{itemize}
    \item Marco è un appassionato di escursioni che ama trascorrere il tempo libero nella natura.
    \item Cerca prodotti affidabili e versatili che si adattino a diverse attività all'aperto.
    \item Desidera una piattaforma intuitiva per esplorare rapidamente nuovi arrivi e offerte speciali.
\end{itemize}

\textbf{Obiettivi:}
\begin{itemize}
    \item Trovare attrezzature versatili per diverse attività all'aperto.
    \item Essere aggiornato sulle nuove offerte e sconti.
    \item Esplorare prodotti in modo rapido ed efficiente.

\end{itemize}

\section{Backend}

\section{Database}
