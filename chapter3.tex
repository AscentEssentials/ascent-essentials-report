\chapter{Design}
% Design dell'architettura del sistema e delle interfacce utente.

\section{Frontend}

% https://www.usability.gov/what-and-why/user-centered-design.html#:~:text=The%20User%2Dcentered%20design%20(UCD,basis%20for%20many%20UCD%20methodologies.

% https://www.interaction-design.org/literature/topics/personas#:~:text=Personas%20are%20fictional%20characters%2C%20which,%2C%20experiences%2C%20behaviors%20and%20goals.

\begin{itemize}
    \item \textbf{Ricerca utente approfondita}: Prima di iniziare il processo di progettazione, è stata condotta una ricerca approfondita sugli utenti target, compresi gli appassionati di arrampicata, alpinismo e trekking. Sono state utilizzate tecniche come interviste, sondaggi e analisi dei dati per comprendere le esigenze, i comportamenti e le preferenze degli utenti.
    \item \textbf{Coinvolgimento attivo degli utenti}: Gli utenti sono stati coinvolti attivamente durante tutto il processo di progettazione attraverso sessioni di feedback, test utente e partecipazione a forum di discussione. Le opinioni degli utenti sono state considerate preziose per le decisioni di design, con regolari sessioni di test per valutare l'usabilità e raccogliere feedback.
    \item \textbf{Creazione di Personas}: Sono state create Personas rappresentative degli utenti target (come Luca l'Alpinista e Sofia la Trekker) per focalizzare la progettazione su utenti specifici con esigenze e obiettivi distinti. Le Personas sono diventate guide chiave per valutare ogni aspetto del design, assicurando che soddisfi le esigenze di utenti reali.
    \item \textbf{Test iterativi}: L'approccio di progettazione ha incorporato cicli iterativi di test utente per identificare e correggere rapidamente eventuali problemi o inefficienze nell'interfaccia utente. Ogni fase del design è stata seguita da test utente, permettendo un'evoluzione continua basata sul feedback reale.
    \item \textbf{Personalizzazione dell'interfaccia}: L'interfaccia utente è stata progettata per essere personalizzabile, consentendo agli utenti di adattare l'esperienza alle proprie preferenze e esigenze. Funzionalità come liste dei desideri personalizzabili, suggerimenti basati su acquisti precedenti e profili utente configurabili sono state implementate per offrire un'esperienza personalizzata.
    \item \textbf{Integrazione continua del feedback}: La creazione di canali dedicati per il feedback degli utenti è stata integrata nell'applicazione, garantendo una comunicazione aperta e continua tra gli sviluppatori e gli utenti. Feedback dai clienti è stato considerato prezioso per guidare gli aggiornamenti e le migliorie continue dell'applicazione.
    \item \textbf{Design responsivo}: L'interfaccia è stata progettata per essere totalmente responsiva, garantendo un'esperienza utente coerente su diverse piattaforme e dispositivi. L'utilizzo di design fluido e layout adattivi assicura che Ascent Essentials sia accessibile su dispositivi mobili, tablet e desktop.
    \item \textbf{Accessibilità universale}: L'attenzione è stata posta sulla progettazione di un'interfaccia accessibile, rispettando le linee guida WCAG per garantire che tutti gli utenti, indipendentemente dalle loro capacità, possano utilizzare l'applicazione. Test automatici e manuali sono stati eseguiti per assicurare che l'applicazione rispetti gli standard di accessibilità.
    \item \textbf{Approccio empatico}: Gli sviluppatori hanno adottato un approccio empatico, cercando di comprendere le emozioni e le prospettive degli utenti per migliorare la qualità dell'esperienza utente complessiva. Le decisioni di design sono state guidate non solo da dati oggettivi, ma anche da una comprensione empatetica delle esigenze e dei desideri degli utenti.
    \item \textbf{Aggiornamenti basati su dati}: Le decisioni di design e gli aggiornamenti futuri sono basati su dati concreti derivati da metriche di utilizzo, feedback degli utenti e analisi delle tendenze di acquisto. L'analisi costante delle metriche di utilizzo e il monitoraggio del comportamento degli utenti guidano gli sviluppatori nella pianificazione di aggiornamenti e miglioramenti mirati.
\end{itemize}

\section{Backend}

\section{Database}
