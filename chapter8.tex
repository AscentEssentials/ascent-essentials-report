\chapter{Conclusioni}
In questo progetto di sviluppo software, abbiamo lavorato con l'obiettivo di creare un E-Commerce specializzato, denominato "AscentEssentials", per fornire attrezzature e accessori agli appassionati delle attività sportive in montagna.

Possiamo dire di aver rispettato tutti i requisiti definiti nelle categorie di requisiti di business, utente, funzionali, non funzionali e di implementazione.

I requisiti utente hanno giocato un ruolo cruciale nella definizione delle funzionalità dell'interfaccia utente. Abbiamo progettato un'interfaccia intuitiva che consente ai clienti di esplorare il catalogo prodotti in modo dettagliato, applicare filtri, utilizzare il carrello e completare il processo di checkout. Per i venditori, l'interfaccia fornisce funzionalità di gestione dei prodotti, degli ordini e dei codici promozionali.

I requisiti di implementazione hanno guidato le scelte tecnologiche e metodologiche. È stato adottato lo stack MEAN per lo sviluppo del sistema. TypeScript è stato scelto per entrambi i lati, backend e frontend, garantendo una gestione più robusta del codice. L'utilizzo di Swagger per la documentazione dell'API del backend ha facilitato la comunicazione tra i membri del team e ha semplificato l'implementazione del frontend.

Siamo soddisfatti dei risultati ottenuti e riteniamo che le competenze acquisite durante questo progetto saranno preziose per il nostro percorso accademico e professionale.