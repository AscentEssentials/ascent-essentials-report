
\chapter{Codice}
% Solo aspetti rilevanti.

\section{Frontend}

Angular, un potente framework front-end, gioca un ruolo centrale nell'implementazione di Ascent Essentials. Con una serie di funzionalità avanzate, Angular offre un ambiente di sviluppo robusto e flessibile, contribuendo a creare un'esperienza utente sofisticata e interattiva.

\subsection{Component}

Angular organizza l'interfaccia utente in \textit{component}, che rappresentano parti indipendenti e riutilizzabili dell'applicazione. Ogni componente ha il proprio template, logica e stile.

\subsection{Service}

I \textit{service} in Angular forniscono una via per organizzare e condividere la logica di business tra diversi \textit{component}. Sono spesso utilizzati per gestire operazioni come il recupero dei dati, l'autenticazione degli utenti e altre funzionalità condivise.

\lstinputlisting[style=JavaScript]{listings/cart.service.ts}

\subsection{Dependency Injection}
Angular utilizza un sistema di Dependency Injection per fornire ai \textit{component} e ai \textit{service} le dipendenze necessarie. Questo approccio semplifica la gestione delle dipendenze e favorisce la modularità e la testabilità del codice.

\subsection{Routing}
Il modulo di routing di Angular consente la navigazione tra diverse parti dell'applicazione senza ricaricare la pagina. Questo è utile per creare un'esperienza utente fluida e per organizzare le diverse sezioni dell'applicazione.

\lstinputlisting[style=JavaScript]{listings/app.routes.ts}

\subsection{Guard}

I \textit{guard} in Angular consentono di controllare l'accesso alle parti specifiche dell'applicazione in base a determinate condizioni, ad esempio, se l'utente è autenticato o se ha determinati ruoli.

\lstinputlisting[style=JavaScript]{listings/user-permission.guard.ts}

\section{Backend}

\subsection{Routes}
Le \textit{routes} in Express consentono di definire le azioni da eseguire quando viene ricevuta una richiesta HTTP.
Come riportato dal seguente codice di esempio, le operazioni sono definite in un controller, che è un file separato che contiene la logica di business:
\lstinputlisting[style=JavaScript]{listings/backend.routes.ts}

\subsection{Middleware}
I \textit{middleware} vengono eseguiti prima di eseguire le operazioni definite nelle \textit{routes}.

\subsubsection{Autenticazione}

\lstinputlisting[style=JavaScript]{listings/backend.auth.ts}

Il \textit{middleware} di autenticazione controlla se nella richiesta è presente un token valido, e in caso affermativo, aggiunge i dettagli dell'utente alla richiesta.

\subsubsection{Autorizzazione}

\lstinputlisting[style=JavaScript]{listings/backend.isAdmin.ts}

Questo \textit{middleware} controlla se l'utente è un amministratore e, in caso affermativo, consente di accedere alla risorsa richiesta.

\subsection{Websocket}
Quando un utente si connette, il suo ID viene associato al socket utilizzato per la connessione. Questo consente di inviare messaggi a utenti specifici.

\lstinputlisting[style=JavaScript]{listings/backend.socket.ts}

