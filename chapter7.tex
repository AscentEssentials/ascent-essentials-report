\chapter{Deployment}
% Rilascio, installazione e messa in funzione.

Per il deployment viene fornito un file Docker Compose, che permette di avviare tutti i servizi necessari al funzionamento dell'applicazione. Il file include un server MongoDB, un pannello di controllo di MongoDB, il servizio backend e il servizio frontend. I servizi sono configurati per comunicare tra loro e le variabili d'ambiente necessarie sono impostate per un corretto funzionamento. Lo storage persistente dei dati è garantito per MongoDB e gli upload di file tramite volumi montati. Il frontend Angular dipende dal backend Node.js e entrambi dipendono dal servizio MongoDB.

Descrizione dettagliata dei servizi presenti nel file \texttt{docker-compose.yml}:

\begin{itemize}
    \item mongo:
    \begin{itemize}
        \item Utilizza l'immagine ufficiale di mongo.
        \item Configura le variabili d'ambiente per il nome utente e la password root di MongoDB.
        \item Collega la porta 27017 all'host.
        \item Monta un volume per lo storage persistente dei dati in ./data.
    \end{itemize}
    \item mongo-dashboard:
    \begin{itemize}
        \item Dipende dal servizio mongo.
        \item Utilizza l'immagine ufficiale di mongo-express.
        \item Collega la porta 8081 all'host per il pannello di controllo di MongoDB.
        \item Configura le variabili d'ambiente per le credenziali di amministrazione di MongoDB e l'autenticazione di base.
        \item Configura l'URL di MongoDB per la connessione al servizio mongo.
    \end{itemize}
    \item backend:
    \begin{itemize}
        \item Dipende dal servizio mongo.
        \item Costruisce un'immagine utilizzando il Dockerfile.backend.
        \item Collega la porta 3000 all'host.
        \item Configura le variabili d'ambiente per la connessione a MongoDB.
        \item Monta un volume per gli upload dei file in ./uploads.
    \end{itemize}
    \item frontend:
    \begin{itemize}
        \item Dipende dal servizio backend.
        \item Costruisce un'immagine utilizzando il Dockerfile.frontend.
        \item Collega la porta 4200 all'host.
        \item Configura le variabili d'ambiente per la connessione a MongoDB.
    \end{itemize}
\end{itemize}

Per tutti i dettagli su come effettuare il deployment dell'applicazione, vengono fornite le istruzioni al seguente link: \url{https://github.com/AscentEssentials/ascent-essentials-docker}
