
\chapter{Requisiti}
% Descrizione delle caratteristiche e funzionalità che il sistema prevede. 

\section{Requisiti di Business}
I requisiti aziendali definiscono gli obiettivi di alto livello e la motivazione dietro lo sviluppo del software. Rispondono alla domanda "perché?" e definiscono l'importanza strategica del progetto.
Gli obiettivi di business includono:
\begin{itemize}
    \item Fornire una piattaforma di e-commerce specializzata nell'equipaggiamento per attività sportive in montagna.
    \item Creare un catalogo prodotti con funzionalità di filtraggio per agevolare la ricerca.
    \item Implementare un sistema di carrello e checkout efficiente.
    \item Offrire la possibilità di registrazione e gestione dell'account utente.
    \item Supportare un account commerciante per la gestione dei prodotti e degli ordini.
    \item Fornire uno storico degli ordini e la possibilità di tracciare lo stato di spedizione.
    \item Implementare un motore di ricerca integrato per facilitare la scoperta di prodotti.
    \item Introdurre notifiche per clienti e commercianti relative al processo di ordine e spedizione.
    \item Gestire codici promozionali durante la fase di checkout.
\end{itemize}

\section{Requisiti Utente}
I requisiti utente si concentrano sull'esperienza dell'utente finale e sulla modalità di interazione con il sistema. Per AscentEssentials, i requisiti utente includono:
\begin{itemize}
    \item Interfaccia utente intuitiva e di facile navigazione.
    \item Registrazione e gestione dell'account utente.
    \item Possibilità di esplorare il catalogo prodotti con filtri intuitivi.
    \item Carrello e processo di checkout user-friendly.
    \item Notifiche chiare e tempestive relative agli ordini e alla spedizione.
\end{itemize}