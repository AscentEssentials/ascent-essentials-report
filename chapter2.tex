
\chapter{Requisiti}
% Descrizione delle caratteristiche e funzionalità che il sistema prevede. 

In questa sezione verranno delineati i vari requisiti del sistema in base alle diverse categorie di requisiti identificate.

\section{Requisiti di Business}
I requisiti aziendali definiscono gli obiettivi di alto livello e la motivazione dietro lo sviluppo del software. Rispondono alla domanda "perché?" e definiscono l'importanza strategica del progetto.

Il progetto "Ascent Essentials" mira a creare un E-Commerce specializzato nella fornitura di attrezzature e accessori per gli appassionati delle attività sportive in montagna.

\section{Requisiti Utente}
I requisiti utente si concentrano sull'esperienza dell'utente finale e sulla modalità di interazione con il sistema.

\begin{enumerate}
    \item Il cliente deve avere accesso a un'interfaccia utente intuitiva con le seguenti funzionalità:
          \begin{enumerate}
              \item Esplorare il catalogo prodotti con dettagli e immagini.
              \item Applicare filtri per la ricerca dei prodotti.
              \item Utilizzare un carrello e completare il processo di checkout.
              \item Registrarsi e gestire l'account utente.
              \item Visualizzare lo storico degli ordini e visualizzare lo stato di spedizione.
              \item Utilizzare un motore di ricerca per i prodotti.
              \item Ricevere notifiche relative al processo di spedizione.
              \item Utilizzare codici promozionali durante il checkout.
          \end{enumerate}
    \item Il venditore deve avere accesso a un'interfaccia utente intuitiva con le seguenti funzionalità:
          \begin{enumerate}
              \item Gestire i prodotti e le categorie.
              \item Gestire gli ordini dei clienti.
              \item Gestire i codici promozionali.
              \item Ricevere notifiche relative a nuovi ordini.
          \end{enumerate}
\end{enumerate}

\section{Requisiti Funzionali}
I requisiti funzionali delineano le funzioni specifiche del sistema, cioè cosa il sistema deve fare:
\begin{enumerate}
    \item Implementazione del catalogo prodotti con dettagli e immagini.
    \item Categorie di prodotti con filtri applicabili.
    \item Funzionalità di carrello e processo di checkout.
    \item Sistema di registrazione e gestione dell'account utente.
    \item Account commerciante per la gestione dei prodotti e degli ordini.
    \item Storico degli ordini e visualizzazione dello stato di spedizione.
    \item Motore di ricerca integrato.
    \item Notifiche per clienti e commercianti.
    \item Gestione di codici promozionali.
\end{enumerate}

\section{Requisiti Non Funzionali}
I requisiti non funzionali definiscono attributi di qualità, vincoli e proprietà generali del sistema:
\begin{enumerate}
    \item Affidabilità: Il sistema deve garantire la corretta gestione degli ordini e delle transazioni, assicurando un'esperienza affidabile per gli utenti.
    \item Performance: Il sistema deve rispondere tempestivamente alle richieste degli utenti, anche in presenza di un grande numero di prodotti e utenti.
    \item Sicurezza: La sicurezza delle informazioni degli utenti e delle transazioni deve essere una priorità, prevenendo accessi non autorizzati e garantendo la privacy.
    \item Interfaccia Utente: L'interfaccia utente deve essere intuitiva, di facile navigazione e compatibile con diversi dispositivi.
    \item Disponibilità: Il sistema deve essere disponibile in modo continuativo per consentire agli utenti di effettuare acquisti in qualsiasi momento.
\end{enumerate}

\section{Requisiti di Implementazione}
I requisiti di implementazione riguardano gli aspetti tecnologici e metodologici dell'implementazione del sistema:
\begin{enumerate}
    \item Utilizzo dello stack MEAN (MongoDB, Express, Angular, Node.js) per lo sviluppo del sistema.
    \item Uso di TypeScript per lo sviluppo sia del backend che del frontend.
    \item Utilizzo di Swagger per la documentazione dell'API del backend.
\end{enumerate}