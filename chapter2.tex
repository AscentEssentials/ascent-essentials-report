
\chapter{Requisiti}
% Descrizione delle caratteristiche e funzionalità che il sistema prevede. 

In questa sezione verranno delineati i vari requisiti del sistema in base alle diverse categorie di requisiti identificate.

\section{Requisiti di Business}
I requisiti aziendali definiscono gli obiettivi di alto livello e la motivazione dietro lo sviluppo del software. Rispondono alla domanda "perché?" e definiscono l'importanza strategica del progetto.

Il progetto "AscentEssentials" mira a creare un E-Commerce specializzato nella fornitura di attrezzature e accessori per gli appassionati delle attività sportive in montagna.

\section{Requisiti Utente}
I requisiti utente si concentrano sull'esperienza dell'utente finale e sulla modalità di interazione con il sistema. Per AscentEssentials, i requisiti utente includono:

\begin{enumerate}
    \item L'utente deve avere accesso a un'interfaccia utente intuitiva con le seguenti funzionalità:
          \begin{enumerate}
              \item Esplorare il catalogo prodotti con dettagli e immagini.
              \item Applicare filtri per la ricerca dei prodotti.
              \item Utilizzare un carrello e completare il processo di checkout.
              \item Registrarsi e gestire l'account utente.
              \item Utilizzare un account commerciante per gestire i prodotti e gli ordini dei clienti.
              \item Visualizzare lo storico degli ordini e visualizzare lo stato di spedizione.
              \item Utilizzare un motore di ricerca per i prodotti.
              \item Ricevere notifiche relative al processo di ordine e spedizione.
              \item Gestire codici promozionali durante il checkout.
          \end{enumerate}
    \item L'utente deve avere la possibilità di aggiungere o rimuovere nuovi prodotti in tempo reale, senza compromettere il funzionamento del sistema.
\end{enumerate}