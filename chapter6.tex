\chapter{Test}
% Test effettuati sul codice e test con utenti.

\section{Frontend}

% https://www.neurowebdesign.it/it/alla-scoperta-delle-10-euristiche-di-nielsen/

La fase di test con gli utenti è essenziale per garantire che Ascent Essentials offra un'esperienza utente ottimale e risponda alle effettive esigenze degli appassionati di attività sportive in montagna. Durante questa fase, abbiamo adottato le euristiche di Nielsen, un set di principi di valutazione dell'usabilità, per valutare in modo sistematico e critico l'interfaccia e l'esperienza complessiva dell'utente. Di seguito sono riportati i risultati dei test, evidenziando le specifiche che sono state prese in considerazione:

\begin{itemize}
    \item \textbf{Visibilità dello Stato del Sistema:} Feedback chiari e tempestivi sono stati implementati per garantire una comprensione immediata dello stato dell'applicazione.
    
    \item \textbf{Corrispondenza tra il Sistema e il Mondo Reale:} Abbiamo cercato di mantenere una corrispondenza intuitiva tra il sistema e il mondo reale.
  
    \item \textbf{Controllo e Libertà dell'Utente:} Gli utenti hanno la possibilità di navigare liberamente all'interno dell'applicazione con percorsi di navigazione chiari.
  
    \item \textbf{Consistenza e Standard:} Abbiamo seguito standard e convenzioni di design coerenti all'interno dell'applicazione per ridurre la curva di apprendimento degli utenti.
  
    \item \textbf{Prevenzione degli Errori:} Sono stati implementati controlli e feedback preventivi per ridurre la probabilità di errori da parte degli utenti.
  
    \item \textbf{Riconoscimento anziché Ricordo:} Abbiamo favorito il riconoscimento piuttosto che il ricordo, rendendo opzioni e azioni visibili e accessibili.
  
    \item \textbf{Flessibilità ed Efficienza di Utilizzo:} L'applicazione è stata progettata per essere flessibile ed efficiente, l'accesso rapido alle informazioni.
  
    \item \textbf{Design ed Estetica Minimalista:} Un design minimalista è stato adottato per concentrare l'attenzione sugli elementi essenziali.
  
    \item \textbf{Aiuto agli Utenti a Riconoscere, Diagnosticare e Recuperare gli Errori:} Messaggi di errore chiari sono stati forniti per aiutare gli utenti a gestire gli errori in modo efficiente.
  
    \item \textbf{Aiuto e Documentazione:} Data la semplicità di utilizzo del sistema, si ritiene che l'utente non abbia bisogno di consultare la documentazione relativa.
  \end{itemize}
  
  I risultati dei test hanno guidato miglioramenti iterativi, contribuendo a una maggiore usabilità e soddisfazione degli utenti.

\section{Backend}
Swagger è utilizzato come strumento per testare il corretto funzionamento del backend. Consente di interagire direttamente con l'API dalla documentazione, rendendo comodo convalidare il comportamento dei diversi percorsi.

La documentazione Swagger viene generata automaticamente in base ai commenti inseriti nel codice, consentendo di mantenere la documentazione aggiornata con il codice.