\chapter{Test}
% Test effettuati sul codice e test con utenti.

\section{Frontend}

% https://www.neurowebdesign.it/it/alla-scoperta-delle-10-euristiche-di-nielsen/

\section{Backend}
Swagger è utilizzato come strumento per testare il corretto funzionamento del backend. Consente di interagire direttamente con l'API dalla documentazione, rendendo comodo convalidare il comportamento dei diversi percorsi.

La documentazione Swagger viene generata automaticamente in base ai commenti inseriti nel codice, consentendo di mantenere la documentazione aggiornata con il codice.